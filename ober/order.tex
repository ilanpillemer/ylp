\documentclass{article}
\usepackage[osf,p]{libertinus}
\usepackage{microtype}
\usepackage[pdfusetitle,hidelinks]{hyperref}
\usepackage[series={},nocritical,noend,nofamiliar,noledgroup]{reledmac}
\usepackage{reledpar}

\usepackage{graphicx}
\usepackage{polyglossia}
\setmainlanguage{english}
\setotherlanguage{hebrew}
\gappto\captionshebrew{\renewcommand\chaptername{קאַפּיטל}}
\usepackage{metalogo}

\linenumincrement*{1}
\firstlinenum*{100000000}
\setlength{\Lcolwidth}{0.44\textwidth}
\setlength{\Rcolwidth}{0.44\textwidth}


\begin{document}
\renewcommand{\abstractname}{\vspace{-\baselineskip}}
\title{
\RLE{
אָבער דאַװנען מוז מען דאָך.
}  
}
\author{by YL Peretz \\ \\ Transl. Ilan Pillemer}

\date{\today}

\maketitle
\abstract{}
\vspace*{-85em}
\begin{pairs}

\begin{Rightside}
\begin{RTL}
\begin{hebrew}
\beginnumbering
\autopar

בערל שנײדער האַט קוים דערלעבט זײן זוהן, דער דאָקטאָר,
זאל צוריק אהײם קומען. אין דער הײם װעט ער האָבען פּראַקטיקע,
מען װעט שוין קרענקען.

געקומען איז דער זוהן פֿרײטאַג, און שבת װיל איהם דער פֿאָטער מיטנעמען דאַװנען.

־־ איך װעל נישט געהן, טאַטע ־־  מאַכט דער דאָקטאָר.
  
 ־־ װאָס, שעמסט דיך מיט מיר צו געהן?

־־ גאָט באַהיט, טאַטע, װאָס פֿאַלט דיר אײן...

־־ אָבער, אז מען איז אַ דאָקטאָר, מײנסט דו, דאַרף מען שוין נישט גאָט בעטען, גאָט לויבען?...

־־ נישט דאָס, טאַטע...

־־ װאָס דען? זאַג! ביסט אפֿשר מיר, חס ושלום, נישט געזונט.

־־ נײן, טאַטע! נאַר געהן װעל איך נישט.

 ־־ איך בין א בעלן צו װיסען פֿאַרװאָס.

־־ נו ,אַדרבאַ, זעץ דיך, טאַטע, איך װעל דיר זאָגען פֿאַרװאָס.

דער אַלטער לעגט אַװעק דעם טלית און זעצט זיך.

־־ נו, זאָג אַדרבה, לאָמיך אויך פֿאַרשטעהן אַ זאַך!

־־ גוט... שטעל דיך פֿאָר, טאַטע, אַז דו ביסט רײך, אַזוי 
רײך, אַז עטליכע רובל מאַכען בײ דיך גאָרנישט אויס...

דער פֿאָטער זיפֿצט אָפּ. כדי פֿון זוהן אַ דאָקטאָר צו מאַכען,
האָט ער זיך איבער'ן קאָפּ פֿאַרזעצט. א הײזעל האָט ער געהאַט ־
אוים הײזעל, קאָמאָרינע זיצט ער, די מאַשינען אויך פֿאַרקויפט.

־־ נו, ־־  מאַכט הער, און זיפֿצט שװער אָפּ.

־־אַלזאָ, דו ביסט רײך, און קעגענאיבער דיך װאוינט אַן אַלמנה, אַ שװאַכע, אַ קראַקנקע אַלמנה,
לאָז נאָך זײן מיט קינדער און מען דאַרף זי שטיצען.

־־ װאַלט איך אודאי געשטיצט!
 
 ־־ נאַר, צי װאַלסטו געװאַרט, ביז די אַלמנה זאָל דיך קומען בעטען.
  חלש'ען פֿאַר דיר,  אויסגיסען ביכען טרערען?

־־ חס־ושלום? נאָר װאָס? אז איך װײס... 

־־ און גאָט ־־ װאָס איז ער, בעסער אָדער ערגער פֿאַר דיר?

־־ װאָס רעדסט דו, א שאלה דאָס איז?

־־ נו, טריומפֿירט דער זוהן דער דאַקטאַר, אז גאָט איז בעסער,
װײס ער אלײן, װאָס אַן אַרעמער, שװאַכער און
קרענקליבערמענטש באַדאַרף, און װעט נישט װאַרטען, מען איהם בעטען...

־־ אָבער...

־־ גאַט לויבען, מײנסט דו?

־־ זאָל זײן...

־־ נו, טאַטע, װי װאַלט דיר געפֿעלען, עס זאַל זיך עמיץ
אַנטקעגען שטעלען און לויבען אין די אויגען אַרײן : אַ װאוילער
שנײדער, אַ גוטער שנײדער, אַן עהרליבער שנײדער, אָבער װאָס
הײסט אַ שנײדער! נאַר אַן אמת'ער שנײדער! אַן אײנאײנציגער
שנײדער!

־־ אך, ־־ מאַכט אוםגעדוליג דער אַלטער, די נרינע גאַל געהט זיך אָן!

־־ און װײסט דו פֿאַרװאָס? װײל דו ביסט קײן נאַר נישט,
זאָלסט הנאָה האָבען פֿון נאַרישען לויבען. און דו ביסט א מענש,
אַ שװאַכער מענש, װאָס שענדען קאָן דיר שאַטען, און לויבען
העלפֿען...

־־ אָבער...

־־ קײן אָבער! טאַטע, קײן אָבער! גאָט איז קליגער פֿאַר
אונז, און ער, מײנסט דו, דאַרף אונזער לויב! ער דאַרף מען זאָל
זיך דרײ מאָל אין טאָג אַנידערשטעלען: אַ גוטער שנײדער, אַ
װאוילער שנײדער ־־

װאָס רעדסט דו... 

־־ נו, זאָל זײן: אַ גוטער, אַ װאוילער גאָט, הימעל
און ערד באַשאַפֿען... צי װײס ער נישט בעסער?

דער אַלטער פֿאַרטראַכט זיך אַ רגע און כאַפּט זיך אינמיטען
אויף:

־־ אַלץ ריכטיג, זאָגט ער, ־־ אָבער ד אַ ו ו ע ן מוז מען דאַרף!  

\endnumbering
\end{hebrew}
\end{RTL}
\end{Rightside}
\begin{Leftside}
\begin{english}
\beginnumbering
\autopar
Berl Shneider had just lived to see his son, the doctor, come back home.
At home, he would have clientele as there were already sick people!

On Friday his son had arrived, and on Shabbos the father wants to take him along with him to pray.

``I will not go, Daddy.", said the doctor.

``What, are you embarrassed to go with me?"

``God forbid, that that should occur to you..."

``But, do you think that if one is a doctor, one no longer needs to entreat God, praise God?"

``Not that, Daddy."

``Then what? Tell me! Can it possibly be, heaven forbid, that we are unwell?"

``No, Daddy. I just do not want to go."

``I am eager to know why."

"Nu, of course, sit down, Daddy, I will tell you why."

The parent put aside his talis, and sat himself down.

``Nu, so out with it, let me also understand the matter." 

``Good, imagine, Daddy that you are a rich, so rich, that a couple of rubels
makes no difference to you..."

The father sighed. In order for his son to become a doctor, he had had to mortgage himself over the head.
He used to have a house, now he is is a tenant; all his equipment he had also sold.

``Go on", he said, and let out a heavy sigh.

``So! Good! You are rich, and opposite you lives a widow, a weak sick widow. Further suppose there
are children, and she needs to be supported."

``Of course, I would have provided support!"

``Now, would you have waited until the widow came begging to you, in weakness before you, with an
outpouring of rivers of tears?"

``God Forbid? And what more? When I know... "

``And God, is he better or worse than you?"

``What are you saying? What kind of question is this?"

``So, triumphed his son, the doctor, then since God is better, and knows himself, what a poor weak and sickly person
needs; and he would not wait until he was entreated..."

``But..."

``You think that we should praise God?"

``One should..."

``Nu, Daddy, how would you like it if someone were to place themselves opposite you and praise you directly
into your eyes : `A great Shneider, a good Shneider, an honest Shneider, but what can one call a Shneider! So truly a
Shneider! An absolutely unique Shneider.' "

``Ach", said the father impatiently, ``such bald foolishness."

``And do you know what? You are no fool that would get pleasure from such foolish praise. And you
are a man, a weak man, and disgrace can harm you and praise help..."

``But...

``No `But'! Daddy, No `But'! God is wiser than us. And do you think that he needs our praise! That he needs one
to stand up erect for him three times a day saying : `A good Shneider, a great Shneider.' "

``What are you saying..."

``Nu, it should be: `A good God, a great God, who created Heaven and Earth....' But surely he knows better?"

The parent is lost in thought for a moment and then in the midst exclaims:

``All true", he says, ``but you still need to P R A Y!." 

\endnumbering
\end{english}
\end{Leftside}
\end{pairs}
\Columns
\end{document}